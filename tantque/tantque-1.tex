\documentclass[11pt,class=report,crop=false]{standalone}
\usepackage[screen]{../python}

\begin{document}


%====================================================================
\chapitre{Arithmétique -- Boucle tant que -- I}
%====================================================================

\objectifs{Les activités de cette fiche sont centrées sur l'arithmétique : division euclidienne, nombres premiers\ldots{} C'est l'occasion d'utiliser intensivement la boucle \og{}tant que\fg{}.}

\insertvideo{Qa9eSV4wB_s}{Arithmétique - Quotient et reste}

\insertvideo{2Xxty4oaHBs}{Boucle tant que I - partie 1}

\insertvideo{NRvv1odnoIU}{Boucle tant que I - partie 2}

\insertvideo{l5Pm1EdxWsQ}{Temps de calcul - Module timeit}


%%%%%%%%%%%%%%%%%%%%%%%%%%%%%%%%%%%%%%%%%%%%%%%%%%%%%%%%%%%%%%%%
%%%%%%%%%%%%%%%%%%%%%%%%%%%%%%%%%%%%%%%%%%%%%%%%%%%%%%%%%%%%%%%%

\begin{cours}[Arithmétique]

\index{division euclidienne}
\index{quotient}
\index{reste}
\index{modulo}
\index{\ci{//}}
\index{\ci{\%}}

On rappelle ce qu'est la division euclidienne. Voici la division de $a$ par $b$, $a$ est un entier positif, $b$ est un entier strictement positif (avec un exemple de $100$ divisé par $7$) :

\myfigure{1}{
\tikzinput{fig-tantque}
}

On a les deux propriétés fondamentales qui définissent $q$ et $r$ :
$$a = b \times q  + r \qquad \text{ et } \qquad 0 \le r < b$$

Par exemple, pour la division de $a=100$ par $b=7$ : on a le quotient $q=14$ et le reste $r=2$ qui vérifient bien $a = b \times q  + r$ car $100 = 7 \times 14 + 2$  et aussi $r<b$ car $2<7$.

\mybox{
Avec \Python{} :
\begin{itemize}
  \item \ci{a // b} \quad renvoie le quotient,
  \item \ci{a \% b} \quad renvoie le reste.
\end{itemize}
}

Il est facile de vérifier que :
\mycenterline{$b$ est un diviseur de $a$ si et seulement si $r=0$.}

\end{cours}


%%%%%%%%%%%%%%%%%%%%%%%%%%%%%%%%%%%%%%%%%%%%%%%%%%%%%%%%%%%%%%%%
% Activité 1
%%%%%%%%%%%%%%%%%%%%%%%%%%%%%%%%%%%%%%%%%%%%%%%%%%%%%%%%%%%%%%%%

\begin{activite}[Quotient, reste, divisibilité]

\objectifs{Objectifs : utiliser le reste pour savoir si un entier divise un autre.}


\begin{enumerate}
  \item Programme une fonction \ci{quotient_reste(a,b)} qui fait les tâches suivantes à partir de deux entiers $a\ge0$ et $b>0$ :
  \begin{itemize}
    \item Elle affiche le quotient $q$ de la division euclidienne de $a$ par $b$,
    \item elle affiche le reste $r$ de cette division,
    \item elle affiche \ci{True} si le reste $r$ est bien positif et strictement inférieur à $b$, et \ci{False} sinon,
    \item elle affiche \ci{True} si on a bien l'égalité $a = bq+r$, et \ci{False} sinon.
   \end{itemize}
    
Voici par exemple ce que doit afficher l'appel \ci{quotient_reste(100,7)}  :
\begin{lstlisting}  
Division de a = 100 par b = 7
Le quotient vaut q = 14
Le reste vaut r = 2
Vérification reste 0 <= r < b ? True
Vérification égalité a = bq + r ? True
\end{lstlisting}

\emph{Remarque.} il faut que tu vérifies sans tricher que l'on a bien $0 \le r<b$ et $a=bq+r$, mais bien sûr cela doit toujours être vrai !
  
  
  \item Programme une fonction \ci{est_pair(n)} qui teste si l'entier $n$ est pair ou pas. La fonction renvoie  \ci{True} ou \ci{False}.
  
  \emph{Indications}
  \begin{itemize}
    \item Première possibilité : calculer \ci{n \% 2} et discuter selon les cas.
    \item Seconde possibilité : calculer \ci{n \% 10} (qui renvoie le chiffre des unités) et discuter.
    \item Les plus malins arriveront à écrire la fonction sur deux lignes seulement (une pour \ci{def...} et l'autre pour \ci{return...})
   \end{itemize}
   
  \item  Programme une fonction \ci{est_divisible(a,b)} qui teste si $b$ divise $a$. La fonction renvoie \ci{True} ou \ci{False}.
  
\end{enumerate}   
     
\end{activite}



%%%%%%%%%%%%%%%%%%%%%%%%%%%%%%%%%%%%%%%%%%%%%%%%%%%%%%%%%%%%%%%%
%%%%%%%%%%%%%%%%%%%%%%%%%%%%%%%%%%%%%%%%%%%%%%%%%%%%%%%%%%%%%%%%
\begin{cours}[Boucle \og{}tant que\fg{}]
La boucle \og{}tant que\fg{} exécute des instructions tant qu'une condition est vraie.
Dès que la condition devient fausse, elle passe aux instructions suivantes.

\index{boucle!tant que}
\index{while@\ci{while}}

\mybox{
\myfigure{0.7}{
  \tikzinput{fig-tantque-cours}
} }

%\begin{minipage}{0.4\textwidth}
%Tant que condition :\\
%\indentation instruction 1\\
%\indentation instruction 2\\
%Instructions suivantes
%\end{minipage}\qquad\qquad
%\begin{minipage}{0.4\textwidth}
%\begin{lstlisting}  
%while condition :
%    instruction 1
%    instruction 2
%Instructions suivantes
%\end{lstlisting}
%\end{minipage}

\begin{exemple}
\begin{minipage}{0.55\textwidth}
Voici un programme qui affiche le compte à rebours $10,9,8,\ldots3,2,1,0$.
Tant que la condition $n \ge 0$ est vraie, on diminue $n$ de $1$. La dernière valeur affichée est $n=0$, car ensuite $n=-1$ et la condition \og{}$n \ge 0$\fg{} devient fausse donc la boucle s'arrête.
\end{minipage}\qquad\qquad
\begin{minipage}{0.4\textwidth}
\begin{lstlisting}
n = 10
while n >= 0:
    print(n)
    n = n - 1
\end{lstlisting}
\end{minipage}

\medskip

On résume ceci sous la forme d'un tableau :
  $$
  \begin{array}{l}
  \text{Entrée : $n=10$}    \\
  \begin{array}{|c|c|c|}
  \hline  
  n & \text{\og $n\ge0$ \fg{} ?} & \text{nouvelle valeur de } n \\
  \hline\hline 
  10 & \text{oui} & 9 \\
  9 & \text{oui} & 8 \\
  \ldots & \ldots & \ldots \\
  1 & \text{oui} & 0 \\
  0 & \text{oui} & -1 \\
  -1 & \text{non} &  \\ 
  \hline
  \end{array} \\
  \text{Affichage : $10,9,8,7,6,5,4,3,2,1,0$}  
  \end{array} 
  $$ 

\end{exemple}


\begin{exemple}
\begin{minipage}{0.55\textwidth}
Ce bout de code cherche la première puissance de $2$ plus grande qu'un entier $n$ donné.
La boucle fait prendre à $p$ les valeurs $2$, $4$, $8$, $16$,\ldots{} Elle s'arrête dès que la puissance de $2$ est supérieure ou égale à $n$, donc ici ce programme affiche $128$.
\end{minipage}\qquad\qquad
\begin{minipage}{0.4\textwidth}
\begin{lstlisting}
n = 100
p = 1
while p < n:
    p = 2 * p
print(p)
\end{lstlisting}
\end{minipage}

  $$
  \begin{array}{l}
  \text{Entrées : $n=100$, $p=1$}    \\
  \begin{array}{|c|c|c|}
  \hline  
  p & \text{\og $p < n$ \fg{} ?} & \text{nouvelle valeur de } p \\
  \hline\hline 
  1 & \text{oui} & 2 \\
  2 & \text{oui} & 4 \\
  4 & \text{oui} & 8 \\
  8 & \text{oui} & 16 \\  
  16 & \text{oui} & 32 \\
  32 & \text{oui} & 64 \\ 
  64 & \text{oui} & 128 \\   
  128 & \text{non} &  \\     
  \hline
  \end{array} \\
  \text{Affichage : $128$}  
  \end{array} 
  $$ 
  
\end{exemple}


\begin{exemple}
\begin{minipage}{0.55\textwidth}
Pour cette dernière boucle on a déjà programmé une fonction \ci{est_pair(n)} qui renvoie \ci{True} si l'entier $n$ est pair et \ci{False} sinon. La boucle fait donc ceci : tant que l'entier $n$ est pair, $n$ devient $n/2$. Cela revient à supprimer tous les facteurs $2$ de l'entier $n$. Comme ici $n = 56 = 2\times 2 \times 2 \times 7$, ce programme affiche $7$.
\end{minipage}\qquad\qquad
\begin{minipage}{0.4\textwidth}
\begin{lstlisting}
n = 56
while est_pair(n) == True:
    n = n // 2
print(n)
\end{lstlisting}



\end{minipage}

  $$
  \begin{array}{l}
  \text{Entrée : $n=56$}    \\
  \begin{array}{|c|c|c|}
  \hline  
  n & \text{\og $n$ est pair \fg{} ?} & \text{nouvelle valeur de } n \\
  \hline\hline 
  56 & \text{oui} & 28 \\
  28 & \text{oui} & 14 \\
  14 & \text{oui} & 7 \\
  7 & \text{non} &  \\ 
  \hline
  \end{array} \\
  \text{Affichage : $7$}  
  \end{array} 
  $$ 
\medskip

Pour ce dernier exemple il est beaucoup plus naturel de démarrer la boucle par
\mycenterline{\ci{while est_pair(n):}}
En effet \ci{est_pair(n)} est déjà une valeur \og{}vrai\fg{} ou \ci{}faux\fg{}.
On se rapproche d'une phrase \og{}tant que $n$ est pair...\fg{}

\end{exemple}


\textbf{Opération \og{}\ci{+=}\fg{}.}
\index{+=@\ci{+=}}
Pour incrémenter un nombre tu peux utiliser ces deux méthodes :
\mycenterline{\ci{nb = nb + 1} \qquad ou \qquad \ci{nb += 1}}
La seconde écriture est plus courte mais rend le programme moins lisible.
\end{cours}



%%%%%%%%%%%%%%%%%%%%%%%%%%%%%%%%%%%%%%%%%%%%%%%%%%%%%%%%%%%%%%%%
% Activité 2
%%%%%%%%%%%%%%%%%%%%%%%%%%%%%%%%%%%%%%%%%%%%%%%%%%%%%%%%%%%%%%%%


\begin{activite}[Nombres premiers]

\objectifs{Objectifs : tester si un entier est (ou pas) un nombre premier.}

\index{diviseur}
\index{nombre premier}

\begin{enumerate}
  \item \textbf{Plus petit diviseur.}
  
  Programme une fonction \ci{plus_petit_diviseur(n)} qui renvoie, le plus petit diviseur $d\ge2$ de l'entier $n\ge2$.
  
  Par exemple \ci{plus_petit_diviseur(91)} renvoie $7$, car $91 = 7 \times 13$.
  
  \medskip
  
  \textbf{Méthode.}
  \begin{itemize}
    \item On rappelle que $d$ divise $n$ si et seulement si \ci{n \% d} vaut $0$.
    \item La mauvaise idée est d'utiliser une boucle \og{}pour $d$ variant de $2$ à $n$\fg{}. En effet, si par exemple on sait que $7$ est diviseur de $91$ cela ne sert à rien de tester si $8,9,10\ldots$ sont aussi des diviseurs car on a déjà trouvé le plus petit.
    
    \item La bonne idée est d'utiliser une boucle \og{}tant que\fg{} !
    Le principe est : \og{}tant que je n'ai pas obtenu mon diviseur, je continue de chercher\fg{}. (Et donc, dès que je l'ai trouvé, j'arrête de chercher.)
    
    \item En pratique voici les grandes lignes :
    \begin{itemize}
      \item Commence avec $d=2$.
      \item Tant que $d$ ne divise pas $n$ alors, passe au candidat suivant ($d$ devient $d+1$).
      \item À la fin $d$ est le plus petit diviseur de $n$ (dans le pire des cas $d=n$).
     \end{itemize} 
  \end{itemize}
  
  
  \item \textbf{Nombres premiers (1).}
  
  Modifie légèrement ta fonction \ci{plus_petit_diviseur(n)} pour écrire une première fonction \ci{est_premier_1(n)} qui renvoie \og{}vrai\fg{} (\ci{True}) si $n$ est un nombre premier et \og{}faux\fg{} (\ci{False}) sinon.
  
  Par exemple \ci{est_premier_1(13)} renvoie \ci{True}, \ci{est_premier_1(14)} renvoie \ci{False}.
  
  \item \textbf{Nombres de Fermat.}
  
  Pierre de Fermat ($\sim$1605--1665) pensait que tous les entiers $F_n = 2^{(2^n)}+1$ étaient des nombres premiers. Effectivement $F_0=3$, $F_1=5$ et $F_2=17$
  sont des nombres premiers. S'il avait connu \Python{} il aurait sûrement changé d'avis ! Trouve le plus petit entier $F_n$ qui n'est pas premier.
  
  \emph{Indication.} Avec \Python{} $b^c$ s'écrit \ci{b ** c} et donc
  $a^{(b^c)}$ s'écrit \ci{a ** (b ** c)}.
  
  \bigskip
  
  \emph{On va améliorer notre fonction qui teste si un nombre est premier ou pas, cela nous permettra de tester plus vite plein de nombres ou bien des nombres très grands.}  
  
  \item \textbf{Nombres premiers (2).}
  
   Améliore ta fonction en une fonction \ci{est_premier_2(n)} qui ne teste pas tous les diviseurs $d$ jusqu'à $n$, mais seulement jusqu'à $\sqrt{n}$.
   
  \emph{Explications.}
  \begin{itemize}
    \item Par exemple pour tester si $101$ est un nombre premier, il suffit de voir s'il admet des diviseurs parmi $2,3,\ldots,10$. Le gain est appréciable !
    
    \item Cette amélioration est due à la proposition suivante : si un entier n'est pas premier alors il admet un diviseurs $d$ qui vérifie $2 \le d \le \sqrt{n}$.
    
    \item Au lieu de tester si $d \le \sqrt{n}$, il est plus facile de tester $d^2 \le n$ !
   \end{itemize} 
  
  
  \item \textbf{Nombres premiers (3).}  
  
  Améliore ta fonction en une fonction \ci{est_premier_3(n)} à l'aide de l'idée suivante.  On teste si $d=2$ divise $n$, mais à partir de $d=3$, il suffit de tester les diviseurs impairs (on teste $d$, puis $d+2$\ldots).
 
 \begin{itemize}
   \item Par exemple pour tester si $n = 419$ est un nombre premier, on teste d'abord si $d=2$ divise $n$, puis $d=3$ et ensuite $d=5$, $d=7$\ldots 
   \item Cela permet de faire environ deux fois moins de tests !
   \item Explications : si un nombre pair $d$ divise $n$, alors on sait déjà que $2$ divise $n$.   
   \end{itemize}
  
  
  \item \textbf{Temps de calcul.}
  
  Compare les temps de calcul de tes différentes fonctions \ci{est_premier()} en répétant par exemple un million de fois l'appel \ci{est_premier(97)}. Voir le cours ci-dessous pour savoir comment faire.
\end{enumerate}   
     
\end{activite}


%%%%%%%%%%%%%%%%%%%%%%%%%%%%%%%%%%%%%%%%%%%%%%%%%%%%%%%%%%%%%%%%
%%%%%%%%%%%%%%%%%%%%%%%%%%%%%%%%%%%%%%%%%%%%%%%%%%%%%%%%%%%%%%%%
\begin{cours}[Temps de calcul]

Il existe deux façons de faire tourner plus rapidement des programmes : une bonne et une mauvaise. La mauvaise, c'est d'acheter un ordinateur plus puissant.
La bonne, c'est de trouver un algorithme plus efficace !

Avec \Python, c'est facile de mesurer le temps d'exécution d'une fonction afin de le comparer avec le temps d'exécution d'une autre. Il suffit d'utiliser le module \ci{timeit}.

\index{module!timeit@\ci{timeit}}
\index{timeit@\ci{timeit}}
\index{temps de calcul}

Voici un exemple : on mesure le temps de calcul de deux fonctions qui ont le même but, tester si un entier $n$ est divisible par $7$.

\begin{lstlisting}  
# Première fonction (pas très maligne)
def ma_fonction_1(n):
    divis = False
    for k in range(n):
        if k*7 == n:
            divis = True
    return divis

# Seconde fonction (plus rapide)
def ma_fonction_2(n):
    if n % 7 == 0:
        return True
    else:
        return False

import timeit

print(timeit.timeit("ma_fonction_1(1000)", 
    setup="from __main__ import ma_fonction_1", 
    number=100000))
print(timeit.timeit("ma_fonction_2(1000)", 
    setup="from __main__ import ma_fonction_2", 
    number=100000))
\end{lstlisting} 


\textbf{Résultats.} \\
Le résultat dépend de l'ordinateur, mais permet la comparaison des temps d'exécution des deux fonctions.
\begin{itemize}
  \item La mesure pour la première fonction (appelée $100\,000$ fois)  renvoie $5$ secondes. L'algorithme n'est pas très malin. On teste
  si $7\times 1 =n$, puis on teste $7\times 2 = n$, $7\times 3 = n$\ldots
  
  \item La mesure pour la seconde fonction renvoie $0.01$ seconde ! On teste si le reste de $n$ divisé par $7$ est $0$. La seconde méthode est donc $500$ fois plus rapide que la première.
\end{itemize}  

\medskip

\textbf{Explications.}
\begin{itemize}
  \item On appelle le module \ci{timeit}.
    
  \item La fonction \ci{timeit.timeit()} renvoie le temps d'exécution en seconde.
Elle prend comme paramètres : 
  \begin{itemize}
    \item une chaîne pour l'appel de la fonction à tester (ici est-ce que $1000$ est divisible par $7$),
    \item un argument \ci{setup="..."} qui indique où trouver cette fonction,
    \item le nombre de fois qu'il faut répéter l'appel à la fonction (ici \ci{number=100000}).
  \end{itemize}
  \item Il faut que le nombre de répétitions soit assez grand pour éviter les incertitudes.
\end{itemize} 

\end{cours}


%%%%%%%%%%%%%%%%%%%%%%%%%%%%%%%%%%%%%%%%%%%%%%%%%%%%%%%%%%%%%%%%
% Activité 3
%%%%%%%%%%%%%%%%%%%%%%%%%%%%%%%%%%%%%%%%%%%%%%%%%%%%%%%%%%%%%%%%

\begin{activite}[Plus de nombres premiers]

\objectifs{Objectifs : programmer davantage de boucles \og{}tant que\fg{} et étudier différentes sortes de nombres premiers à l'aide de ta fonction \ci{est_premier()}.}

\begin{enumerate}
  \item Écris une fonction \ci{nombre_premier_apres(n)} qui renvoie le premier nombre premier $p$ supérieur ou égal à $n$. 
  
  Par exemple, le premier nombre premier après $n=60$ est $p=61$. 
  Quel est le premier nombre premier après $n=100\,000$ ?
  
  \item Deux nombres premiers $p$ et $p+2$ sont appelés \defi{nombres premiers jumeaux}. Écris une fonction \ci{nombres_jumeaux_apres(n)} qui renvoie le premier couple $p$, $p+2$ de nombres premiers jumeaux, avec $p \ge n$.
  
  Par exemple, le premier couple de nombres premiers jumeaux après $n=60$ est $p=71$ et $p+2=73$.   
    Quel est le premier couple de nombres premiers jumeaux après $n=100\,000$ ?
    
  \item Un entier $p$ est un \defi{nombre premier de Germain} si $p$ et $2p+1$ sont des nombres premiers. Écris une fonction \ci{nombre_germain_apres(n)} qui renvoie le couple $p$, $2p+1$ où $p$ est le premier nombre premier de Germain $p \ge n$.
  
  Par exemple, le premier nombre premier de Germain après $n=60$ est $p=83$ avec $2p+1=167$.   
  Quel est le premier nombre premier de Germain après $n=100\,000$ ?
  
\end{enumerate}   
     
\end{activite}


\end{document}
