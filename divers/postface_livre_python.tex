
\clearemptydoublepage
\pagestyle{empty}\thispagestyle{empty}

\vspace*{\fill}

\section*{Remerciements}


Je remercie Stéphanie Bodin pour ses encouragements et pour avoir testé les activités des premières parties.
Je remercie vivement Michel Bodin pour avoir testé toutes les premières activités et sa relecture.
Merci à François Recher pour son enthousiasme et sa relecture attentive du livre. 
Merci à Éric Wegrzynowski pour sa relecture et ses idées pour les activités sur les images. 
Merci à Philippe Marquet pour sa relecture des premiers chapitres. Merci à Kroum Tzanev pour la maquette du livre.


\bigskip

\begin{center}
Vous pouvez récupérer l'intégralité des codes \Python{} des activités ainsi que tous les fichiers sources sur la page \emph{GitHub} d'Exo7 :
\href{https://github.com/exo7math/python1-exo7}{\og{}GitHub : Python au lycée\fg{}}.

\medskip

Les vidéos des cours avec des explications pas à pas et la présentation des projets sont disponibles depuis la chaîne \emph{Youtube} :
\href{https://www.youtube.com/channel/UC6PiFyqBiUjiJ7Q3DRSW2Wg}{\og{}Youtube : Python au lycée\fg{}}.


%\\ \centerline{
%\href{https://www.youtube.com/c/PythonAuLycee}{youtube.com/PythonAuLyce}}
\end{center}


\vspace*{\fill}

\bigskip 

\begin{center}
\LogoExoSept{3}
\end{center}



\begin{center}
Ce livre est diffusé sous la licence \emph{Creative Commons -- BY-NC-SA -- 4.0 FR}.

Sur le site Exo7 vous pouvez télécharger gratuitement le livre en couleur.
\end{center}




\pagenumbering{gobble} % remove page numbering 
\printindex
\pagenumbering{arabic}

